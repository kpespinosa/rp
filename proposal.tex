\documentclass[12pt]{article}

\usepackage{graphics}
\usepackage{epsfig}
\usepackage{times}
\usepackage{amsmath}
\usepackage{url}

% <http://psl.cs.columbia.edu/phdczar/proposal.html>:
%
% The standard departmental thesis proposal format is the following:
%        30 pages
%        12 point type
%        1 inch margins all around = 6.5   inch column
%        (Total:  30 * 6.5   = 195 page-inches)
%
% For letter-size paper: 8.5 in x 11 in
% Latex Origin is 1''/1'', so measurements are relative to this.

\topmargin      0.0in
\headheight     0.0in
\headsep        0.0in
\oddsidemargin  0.0in
\evensidemargin 0.0in
\textheight     9.0in
\textwidth      6.5in

\title{{\bf Special Problem Proposal} \\
\it Semantic Normalization in Unstructured Data}

\author{ {\bf Kurt Junshean Espinosa}  \\
{\small kpespinosa@up.edu.ph}
}
\date{\today\\v1.0}

\begin{document}
\pagestyle{plain}
\pagenumbering{roman}
\maketitle



\pagebreak
\tableofcontents
\pagebreak

\cleardoublepage
\pagenumbering{arabic}

\section{The Normalization Problem}
\label{ch:intro}
Describe the problem here in words and in mathematical notations as well as you can. 

\section{Problem Representations, Approaches and Evaluation}
\label{ch:proposal}
Describe here how the problem is being represented by different researches and the corresponding approaches, results, and evaluation.

\section{Gaps and Challenges}
\label{ch:proposal}
Describe here the challenges and gaps that still remains about the problem.

\section{Proposed Approach}
\label{ch:proposal}
Describe here proposal on how to address the gaps and challenges. You can present here how you model the problem and consequently show how it can be solved generally. Describe why your proposed solution follows intuition. You will present also here the detailed methodology.

\section{Initial Experiments}
Here will be the set of experiments that you will do. Describe here the dataset. Use references like this\cite{Etzioni200591}

\bibliography{proposal} 
\bibliographystyle{apalike}

\end{document}
